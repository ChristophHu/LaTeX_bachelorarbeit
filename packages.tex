% packages ---------------------------------------------------------------------
% Die Datei packages.tex beinhaltet alle benötigten packages alsauch diese,
% welche später verfügbar sein sollten.
% ------------------------------------------------------------------------------

% Seitenstil -------------------------------------------------------------------
\usepackage[
    automark,                           % Kapitelangaben in Kopfzeile automatisch erstellen
    headsepline,                        % Trennlinie unter Kopfzeile
    ilines                              % Trennlinie linksbündig ausrichten
]{scrpage2}

\usepackage[ngerman]{babel}             % ...
\usepackage{moreverb}                   % ...
\usepackage{ae}                         % Schönere Schriften zur PDF-Erstellung

% Schrift ----------------------------------------------------------------------
\usepackage{lmodern}                    % neue deutsche Rechtschreibung zur Silbentrennung,
                                        % korrekten Benennung von Inhaltsverzeichnis und Literaturverzeichnis
\usepackage{relsize}                    % Schriftgröße relativ festlegen

% Umlaute ----------------------------------------------------------------------
\usepackage{ngerman}
\usepackage[latin1]{inputenc}
\usepackage[T1]{fontenc}                % zur korrekten Darstellung von öüä

% Zeichensätze -----------------------------------------------------------------
\usepackage{textcomp}                   % \textcopyright, \texttrademark
\usepackage[official,right]{eurosym}    % EUR-Symbol; \euro oder \EUR{XX,XX}

% Grafiken ---------------------------------------------------------------------
\usepackage[dvips,final]{graphicx}      % Einbinden von JPG-Grafiken ermöglichen
\graphicspath{{ba_logo/}}               % hier liegen die Bilder des Dokuments
\usepackage{floatflt}                   % umfließen von Bildern

% Index-Ausgabe mit \printindex ------------------------------------------------
\usepackage{makeidx}

% Zeilenabstände und Seitenränder etc. -----------------------------------------
\usepackage{setspace}                   % Definition von Zeichen-/Zeilenabständen
\usepackage{geometry}                   % Definition von Seitenrändern

% Index & Hyperlinks -----------------------------------------------------------
\usepackage{makeidx}
\usepackage[colorlinks=true, linkcolor=blue]{hyperref}
%\usepackage{hyperref}                   % genutzt für den Glossar
\usepackage{url}                        % URL verlinken

% Glossar - https://en.wikibooks.org/wiki/LaTeX/Glossary -----------------------
\usepackage{glossaries}                 % zur Erstellung eines Glossars

% Zitierweise ------------------------------------------------------------------
\usepackage[numbers]{natbib}            % Zitierweise, ohne options wird ein Fehler bei fehlendem Jahr im Literaturverzeichnis angezeigt

%\usepackage[
%    style=authoryear,           % Zitierstil
%    isbn=false,                 % ISBN nicht anzeigen, gleiches geht mit nahezu allen anderen Feldern
%    pagetracker=true,           % ebd. bei wiederholten Angaben (false=ausgeschaltet, page=Seite, spread=Doppelseite, true=automatisch)
%    maxbibnames=50,             % maximale Namen, die im Literaturverzeichnis angezeigt werden (ich wollte alle)
%    maxcitenames=3,             % maximale Namen, die im Text angezeigt werden, ab 4 wird u.a. nach den ersten Autor angezeigt
%    autocite=inline,            % regelt Aussehen für \autocite (inline=\parancite)
%    block=space,                % kleiner horizontaler Platz zwischen den Feldern
%    backref=true,               % Seiten anzeigen, auf denen die Referenz vorkommt
%    backrefstyle=three+,        % fasst Seiten zusammen, z.B. S. 2f, 6ff, 7-10
%    date=short,                 % Datumsformat
%    ]{biblatex}
%\setlength{\bibitemsep}{1em}    % Abstand zwischen den Literaturangaben
%\setlength{\bibhang}{2em}       % Einzug nach jeweils erster Zeile
%\bibliography{literature}       % Bibtex-Datei wird schon in der Preambel eingebunden

% abgelegte packages -----------------------------------------------------------
%\usepackage{typearea}           % Fehler zur Präambel-Head-Zeile umgehen


