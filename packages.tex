% packages ---------------------------------------------------------------------
% Die Datei packages.tex beinhaltet alle benötigten packages alsauch diese,
% welche später verfügbar sein sollten.
% ------------------------------------------------------------------------------

% Seitenstil -------------------------------------------------------------------
\usepackage[
    automark,                           % Kapitelangaben in Kopfzeile automatisch erstellen
    headsepline,                        % Trennlinie unter Kopfzeile
    ilines                              % Trennlinie linksbündig ausrichten
]{scrpage2}

\usepackage[ngerman]{babel}             % ...
\usepackage{moreverb}                   % ...
\usepackage{ae}                         % Schönere Schriften zur PDF-Erstellung

% Schrift ----------------------------------------------------------------------
\usepackage{lmodern}                    % neue deutsche Rechtschreibung zur Silbentrennung,
                                        % korrekten Benennung von Inhaltsverzeichnis und Literaturverzeichnis
\usepackage{relsize}                    % Schriftgröße relativ festlegen

% Umlaute ----------------------------------------------------------------------
\usepackage[ngerman]{babel}
\usepackage[latin1]{inputenc}
\usepackage[T1]{fontenc}                % zur korrekten Darstellung von öüä

% Zeichensätze -----------------------------------------------------------------
\usepackage{textcomp}                   % \textcopyright, \texttrademark
\usepackage[official,right]{eurosym}    % EUR-Symbol; \euro oder \EUR{XX,XX}

% Grafiken ---------------------------------------------------------------------
%\usepackage[dvips,final]{graphicx}      % Einbinden von JPG-Grafiken ermöglichen
\usepackage{graphicx}
%\graphicspath{ba_logo/}                 % hier liegen die Bilder des Dokuments
\usepackage{floatflt}                   % umfließen von Bilderna

% Index-Ausgabe mit \printindex ------------------------------------------------
\usepackage{makeidx}

% Zeilenabstände und Seitenränder etc. -----------------------------------------
\usepackage{setspace}                   % Definition von Zeichen-/Zeilenabständen
\usepackage{geometry}                   % Definition von Seitenrändern

% Index & Hyperlinks -----------------------------------------------------------
\usepackage{makeidx}
\usepackage[colorlinks=false]{hyperref}
\usepackage{url}                        % URL verlinken

% Glossar - https://en.wikibooks.org/wiki/LaTeX/Glossary -----------------------
\usepackage{glossaries}                 % zur Erstellung eines Glossars

% Zitierweise ------------------------------------------------------------------
%\usepackage[numbers]{natbib}            % Zitierweise, ohne options wird ein Fehler bei fehlendem Jahr im Literaturverzeichnis angezeigt
%\usepackage{filecontents}               % dient der Erstellung des Literaturverzeichnis durch ba_literaturverzeichnis.tex
\usepackage[backend=biber,style=authoryear]{biblatex}
\bibliography{bachelorarbeit_lit.bib} 
\usepackage[babel,german=quotes,threshold=3]{csquotes}

% Abbildungsverzeichnis --------------------------------------------------------
\usepackage{acronym}

% abgelegte packages -----------------------------------------------------------
%\usepackage{typearea}           % Fehler zur Präambel-Head-Zeile umgehen
\usepackage{bookmark}