\markboth{Erstes Kapitel}{Erstes Kapitel}
\chapter{Erstes Kapitel}                % 1.    Erstes Kapitel
\section{Erstes}                        % 1.1.  Erstes Unterkapitel
\subsection{Zweites Unterkapitel}       % 1.1.1 Zweites Unterkapitel

Affe\index{Affe}
Bieber\index{Bieber}

Das erste Zitat\footfullcite{aristotle:physics}\par
So funktioniert auch das zweite Zitat\footcite{aristotle:physics}\par
So funktioniert auch das zweite 2. Zitat \cite{aristotle:physics}\par
So funktioniert auch das zweite 3. Zitat \autocite[80]{aristotle:physics}\par 
So funktioniert das dritte Zitat leider nicht \autocite{aristotle:physics}\par
Test \citeyear{aristotle:physics} by \citeauthor{aristotle:physics}

Das K\"urzel EU \ac{EU} wird in erster Verwendung auch in der Langform dargestellt.
Wird das gleiche K\"urzel erneut aufgerufen (\ac{EU}) so erfolgt lediglich die Kurzform. % acs gibt lediglich die Kurzform, acl die Langform wieder

\subsection{Weiteres Unterkapitel}       % 1.1.2
\section{Zweites}
\newpage
\chapter{Zweites Kapitel}
\section{Erstes vom Zweiten}