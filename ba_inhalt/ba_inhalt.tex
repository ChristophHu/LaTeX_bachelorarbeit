\markboth{Erstes Kapitel}{Erstes Kapitel}
\chapter{Erstes Kapitel}                % 1.    Erstes Kapitel
\section{Erstes}                        % 1.1.  Erstes Unterkapitel
\subsection{Zweites Unterkapitel}       % 1.1.1 Zweites Unterkapitel

% Test des Index - Einträge erhalten einen Eintrag im Index
Affe\index{Affe}
Bieber\index{Bieber}

% Test zu Zitaten - Eintrag muss im Literaturverzeichnis vorhanden sein
Das erste Zitat\footfullcite{aristotle:physics}\par                                 % gute Variante (Fußzeile, umfänglich)
So funktioniert auch das zweite Zitat\footcite{aristotle:physics}\par               % gute Variante (Fußzeile, kurz/knapp)
So funktioniert auch das zweite 2. Zitat \cite{aristotle:physics}\par               % keine gute Variante
So funktioniert auch das zweite 3. Zitat \autocite[80]{aristotle:physics}\par       % gute Variante (ohne Fußzeile, Name, Jahr, Seite)
So funktioniert das dritte Zitat leider nicht \autocite{aristotle:physics}\par      % gute Variante (ohne Fußzeile, Name, Jahr)
Test \citeyear{aristotle:physics} by \citeauthor{aristotle:physics}                 % Nutzung einzelner Attribute (Jahr, Autor)

% Test von Abkürzungen - Abkürzung muss im Abkürzungsverzeichnis vorh. sein
Das K\"urzel EU \ac{EU} wird in erster Verwendung auch in der Langform dargestellt.
Wird das gleiche K\"urzel erneut aufgerufen (\ac{EU}) so erfolgt lediglich die Kurzform.

% Test der Abbildungsbereschreibung - Abbildungsverzeichnis wird automatisch erstellt
\begin{figure}
    \centering
    \captionof{figure}{Eine Beispielabbildung}    % \caption[Beispielabbildung]{Eine Beispielabbildung}
\end{figure}

% Test der Tabellenbeschreibung - Tabellenverzeichnis wird automatisch erstellt
\newpage
\begin{table}
    \centering
    \captionof{table}{Tabellenbeschreibung}
\end{table}

\subsection{Weiteres Unterkapitel}       % 1.1.2
\section{Zweites}
\newpage
\chapter{Zweites Kapitel}
\section{Erstes vom Zweiten}

\newpage